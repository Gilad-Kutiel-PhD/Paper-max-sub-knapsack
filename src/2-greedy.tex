We consider the algorithm that chooses the best solution between the greedy
solution and the best of all pairs.
We assume w.l.o.g. that $f(O) = 1$, where $O$ is an optimal set.

Let $\varepsilon > 0$ a constant to be determined latter on, and let $S$ be the output of 
the above algorithm.
\begin{theorem}
$f(S) \geq 1 - e^{-(1/2 + \varepsilon)}$
\end{theorem}

\begin{proof}
Fix an optimal solution $O$, 
and let $A$ be the set of elements chosen by the greedy algorithm 
until the first element from the optimal solution, $x$, was dropped. 
If $f(A) \geq 1 - e^{-(1/2 + \varepsilon)}$ we are done, otherwise define 
$f(A) = 1 - e^-{1/2 + \varepsilon - \delta}$. 
This implies that the cost of $A$ is at most $1/2 + \varepsilon - \delta$, and, thus, the 
cost of $x$ is at least $1/2 - \varepsilon + \delta$.
Call an element \emph{heavy} if it costs at least $1/4 + \varepsilon/2 - \delta/2$
(otherwise call it \emph{light}), 
and observe that there are at most 2 heavy elements in the optimal solution.
If the value of the heavy elements of the optimal solution 
is at least $1-e^{-(1/2 + \varepsilon)}$ we are done again. 
Otherwise, the value of the light elements in the optimal solution 
is at least $e^{-(1/2 + \varepsilon)}$.
Also the total cost of the light elements in the optimal solution 
is at most $1/2 + \varepsilon - \delta$.
Thus, the total value obtained by the algorithm is at least the minimum between:
$$
\min_\delta
1-e^{-(1/2 + \varepsilon - \delta)}
+
\left[
e^{-(1/2 + \varepsilon)}
-
(1-e^{-(1/2 + \varepsilon - \delta)})
\right]
(1-e^{-\frac{1-6\varepsilon + 6\delta}{2 + 4\varepsilon - 4\delta}})
$$ 
and
$$
1-e^{-(1/2 + \varepsilon)}
$$
Defining $\varepsilon = 1.04$ completes the proof. 
\end{proof}
