\nonstopmode
\documentclass[a4paper,UKenglish,cleveref, autoref]{lipics-v2019}

\usepackage{amsmath,amssymb,amsthm,mathtools}
\usepackage[linesnumbered, ruled]{algorithm2e}
\usepackage[final]{microtype}
\usepackage[final]{hyperref}
\usepackage[inline]{enumitem}
\usepackage{subcaption}

% ERR FIX
\usepackage[T1]{fontenc}  


% DEBUG
\usepackage{lipsum}
\usepackage{fullpage}
\usepackage{lineno}
\linenumbers

% EXTRA
\usepackage{authblk}

\DeclareMathOperator*{\argmin}{arg\,min}
\DeclareMathOperator*{\argmax}{arg\,max}

\def\R{\mathbb{R}}
\def\N{\mathbb{N}}

\newtheorem{observation}{Observation}
\newtheorem{lemma}{Lemma}
\newtheorem{theorem}{Theorem}

\newcommand{\defeq}{\vcentcolon=}

\newcommand\todo[1]{\textcolor{red}{TODO }{#1}}
% \renewcommand\todo[1]{}

\title{A Fast and Simple Algorithm for Submodular Maximization with a Knapsack Constraint}
\titlerunning{A Fast and Simple Algorithm for Submodular Knapsack}

\author{Ariel Kulik}{Department of Computer Science, Technion, Haifa, Israel}{kulik@cs.technion.ac.il}{}{}
\author{Gilad Kutiel}{Department of Computer Science, Technion, Haifa, Israel}{gkutiel@cs.technion.ac.il}{}{}
\author{Roy Schwartz}{Department of Computer Science, Technion, Haifa, Israel}{schwartz@cs.technion.ac.il}{}{}
\authorrunning{A.\,Kulik and G.\,Kutiel and R.\,Schwartz}

\Copyright{Ariel Kulik and Gilad Kutiel and Roy Schwartz}
\ccsdesc{}

%\ccsdesc[100]{ {\color{red}{TBD}} }
%\ccsdesc[100]{ {\color{red}{TBD}} }

\ccsdesc[100]{Theory of computation~Submodular optimization and polymatroids}
\ccsdesc[100]{Theory of computation~Approximation algorithms analysis}

\keywords{knapsack, submodular function, approximation algorithm}

%\category{}

%\relatedversion{}

%\supplement{}

\nolinenumbers



\begin{document}
\maketitle

\begin{abstract}
We consider the problem of maximizing a monotone submodular function with a knapsack constraint.
In this work we aim at finding fast and simple algorithms for the problem.
Previously, the best fast and simple algorithm is that of Khuller {\em et. al.} [IPL`99] which runs in time $O(n^2)$ and achieves an approximation guarantee of $(1-e^{-\nicefrac[]{1}{2}})\approx 0.393 $.
We present a new fast and simple algorithm which retains the same running time and has an improved approximation guarantee of $0.4536$.
%At the heart of our analysis lies a method that enables us to better analyze the greedy ``bang per buck'' algorithm in the presence of elements with varying cost.
Moreover, we present a general method for ``amplifying'' the approximation factor of any algorithm for the problem, while losing little in the constants of the running time.
Applying this amplification to our new algorithm enables us to further improve the results obtained.
We believe that our amplification method might be of independent interest.

\end{abstract}

\section{Introduction}
Submodularity is a fundamental mathematical notion that captures the concept of economy of scale and is prevalent in many areas of science and technology.
Given a ground set $U$ a set function $f:2^U \to \mathbb{R}_+$ over $U$ is called \emph{submodular} if it has the \emph{diminishing returns} property:
$f(A \cup \{a\}) - f(A) \geq f(B \cup \{a\}) - f(B)$ for every $A \subseteq B \subseteq U$ and $a \in U \setminus B$.\footnote{
    An equivalent definition is: $f(A) + f(B) \geq f(A \cup B) + f(A \cap B)$ for every $A,B \in U$.
}
Submodular functions naturally arise in different disciplines such as combinatorics, graph theory, probability, game theory, and economics.
Some well known examples include coverage functions, cuts in graphs and hypergraphs, matroid rank functions, entropy, and budget additive functions.
Additionally, submodular functions play a major role in many real world applications, {\em e.g.}, the spread of influence in networks \cite{KKT03,KKT05,KKT15,MR10}, recommender systems \cite{EG11,EVSG09}, document summarization \cite{DKR13,LB10,LB11}, and information gathering \cite{GKS05,KG11,KGGK06,KGGK11,KSG08}, are just a few such examples.

Combinatorial optimization problems with a submodular objective have been the focus of intense research in the last decade as such problems provide a unifying framework that captures many fundamental problems in the theory of algorithms and numerous real world practical applications.
Examples of the former include, {\em e.g.}, Max-CUT and Max-DiCUT \cite{FG95,GW95,HZ01,H01,K72,KKMO07,LLZ02,TSSW00}, Max-$k$-Coverage \cite{F98,SW11,V01}, Max-Bisection \cite{ABG13,FJ97,HZ02,Y01}, Generalized-Assignment \cite{CK05,CKR06,FGMS06,FV06}, and Max-Facility-Location \cite{AS99,CFN77a,CFN77b}\footnote{Many of the above mentioned problems can also be found in introductory books to approximation algorithms \cite{SW11,V01}.}, whereas examples of the latter include, {\em e.g.}, pollution detection \cite{KLGVF08}, gang violence reduction \cite{SSPB14}, outbreak detection in networks \cite{LKGFFVG07}, exemplar based clustering \cite{GK10}, image segmentation \cite{KXFK11}, and recommendation diversification \cite{YG11}.

A main driving force behind the above research is the need for algorithms that not only provide provable approximation guarantees, but are also fast and  simple to implement in practice.
This need stems from the sheer scale of the applicability of submodular maximization problems in diverse disciplines, and is further amplified by the fact that many of the practical applications arise in areas such as machine learning and data mining where massive data sets and inputs are ubiquitous.\footnote{Refer to the recent book \cite{B13} and survey \cite{KG14} for additional examples and applications of submodularity in machine learning.}



In this paper we consider the problems of maximizing a monotone\footnote{
    $f$ is monotone if $f(S) \leq f(T)$ for every $S \subseteq T \subseteq U$.
} submodular function given a knapsack constraint.
%{\color{red}{ROY: ADD APPLICATIONS SPECIFIC TO OUR PROBLEM HERE WITH A SHORT EXPLANATION}}.
In this problem we are given a ground set
$U$ of size $n$, a monotone submodular function $f:2^U \to \mathbb{R}_+$, a cost function $c:U \to \mathbb{R}_+$, and a budget $\beta$.
The goal is to find a subset of elements $S$ that maximizes $f(S)$ such that the total cost of the elements in $S$ does not exceed the budget, {\em i.e.}, $\sum _{x\in S}c(x)\leq \beta$.
For abbreviation we denote this problem by \SK.
Besides being a natural problem on its own right, capturing the classic Knapsack problem, \SK admits practical applications, {\em e.g.}, entropy maximization in graphical models \cite{krause2005note}, and document summarization \cite{LB10}.
In this work we aim to find {\em fast} and {\em simple} algorithms for the \SK problem.

We assume the standard value oracle model, where the algorithm can access the objective $f$ via queries of the form: ``what is $f(S)$?'' for every $S\subseteq U$.
The running time is the total number of value oracle queries and numerical operations performed by the algorithm. In all our algorithms the former dominates the later, and thus we are satisfied with counting the number of oracle queries alone.

Building upon the work of Khuller {\em et. al.} \cite{khuller1999budgeted}, Sviridenko \cite{sviridenko2004note} presented a tight approximation of $(1-\nicefrac[]{1}{e})\approx 0.632$ for \SK.
The algorithm of \cite{sviridenko2004note} returns the best of all subsets of $U$ of size at most three, where each subset of size three is greedily extended by the standard greedy rule that maximizes the ``bang per buck''.
This results in an impractical algorithm whose running time is $O(n^5)$.
A fast algorithm whose running time is just that of the greedy algorithm, {\em i.e.}, $O(n^2)$, was given by \cite{khuller1999budgeted} and it achieves a worse approximation of $(1-e^{-\nicefrac[]{1}{2}})\approx 0.393$.\footnote{Khuller {\em et. al.} \cite{khuller1999budgeted} considered the special case where the objective is a coverage function. This was later extended by Krause and Guestrin \cite{krause2005note} and Lin and Bilmes \cite{LB10} to a general monotone submodular objective.}


Deviating from the above combinatorial approach of \cite{khuller1999budgeted,sviridenko2004note} to \SK, Badanidiyuru and Vondr\'{a}k \cite{badanidiyuru2014fast} initiated a different line of research based on both continuous and discrete techniques.
They presented an algorithm achieving an approximation of $(1-\nicefrac[]{1}{e}-\varepsilon)$ whose running time is $O(n^2(\varepsilon ^{-1}\log n)^{\text{poly}(\varepsilon^{-1})})$, for every constant $\varepsilon >0$.
%In contrast to the combinatorial approach of \cite{khuller1999budgeted,sviridenko2004note}, Badanidiyuru and Vondr\'{a}k based their algorithm on a continuous approach.
Building upon the approach of \cite{badanidiyuru2014fast}, Ene and Nguy\~{\^{e}}n \cite{Alina2017} presented an algorithm achieving the same approximation guarantee whose running time it $O(\varepsilon^{-O(\varepsilon^{-4})}n \log^2 n)$.
Both algorithms of \cite{badanidiyuru2014fast,Alina2017} are theoretically interesting and appealing, as they require the introduction of novel ideas that enable one to extrapolate between the discrete and continuous approaches.
Unfortunately, both are not simple nor fast, as even stated by the authors themselves, {\em e.g.}, see \cite{Alina2017}.
To best exemplify the impracticality of these algorithms one needs only to choose $\varepsilon = \nicefrac[]{1}{4}$.
This results in an approximation of $(1-\nicefrac[]{1}{e}-\nicefrac[]{1}{4})$, which is worse than the fast algorithm of \cite{khuller1999budgeted} since $(1-\nicefrac[]{1}{e}-\nicefrac[]{1}{4})<(1-e^{-\nicefrac[]{1}{2}})$, and an impossible running time of at least $2^{512} n$ \cite{Alina2017}.
%This renders the algorithms of \cite{Alina2017,badanidiyuru2014fast} theoretically interesting an appealing, but completely useless.

%Maximizing a monotone submodular function under a knapsack constraint generalized the budgeted maximum coverage problem~\cite{khuller1999budgeted} and has applications such as document summarization~\cite{lin2010multi} and maximizing entropy in discrete graphical models~\cite{krause2005note}.


%\paragraph*{Previous Work}
%Nemhauser et al. considered the problem of maximizing a monotone submodular function under cardinality constant \cite{Nemhauser1978}.
%They proved that the greedy algorithm (one that iteratively construct a solution by selecting each time the best element) gives $(1 - e^{-1})$ approximation.
%Khuller et. al considered the Budgeted Maximum Coverage problem\cite{khuller1999budgeted}.
%A coverage objective is a special case of submodular objective.
%They gave a $(1-e^{-1})$-approximation algorithm for this problem and showed that this is the best possible unless P = NP.
%Sviridenko \cite{sviridenko2004note} showed that the same algorithm presented by
%Khuller et. al can be used to maximize a general monotone submodular function
%with the same guarantee.
%This algorithm requires $O(n^5)$ calls to the value oracle and might be impractical for real world applications~\cite{lin2010multi}.
%
%A faster algorithm that runs in $O(n^2)$ time and achieves a $1 - e^{-1/2}$ approximation ratio was also presented in \cite{khuller1999budgeted}.
%It was shown by Krause and Guestrin \cite{krause2005note} that the same algorithm
%gives the same guarantee for a general monotone submodular function and requires only $O(n^2)$ calls to the value oracle.
%
%Badanidiyuru and Vondr´ak developed a $1 - \frac{1}{e} - \epsilon$-approximation
%algorithms that runs in
%$O(n^2(\frac{1}{\epsilon}\log n)^\text{poly}(\frac{1}{\epsilon}))$ time
%\cite{badanidiyuru2014fast}.
%Ene and Nguyen developed an even faster algorithm that runs in $\frac{1}{\epsilon}^{O(1/\epsilon^4)}n \log^2 n$ time \cite{Alina2017}.
%These algorithms, however, as mentioned by the authors, are impractical.
%For example, the ng time of the latter algorithm for $\epsilon = 2^{-2}$ is
%$2^{2O(2^{8})}n\log^2n$ achieving approximation ratio of $\approx 0.38$.

\paragraph*{Our Results}
Our results can be partitioned into two parts: $(1)$ fast and simple combinatorial algorithms for \SK; and $(2)$ a general method for ``amplifying'' any algorithm for \SK by improving its approximation factor while losing little in the running time.

\noindent {\bf{Fast and Simple Combinatorial Algorithms:}} We note that the proof that the fast algorithm of \cite{khuller1999budgeted} (named Modified Greedy by the authors) achieves an approximation guarantee of $(1-e^{-\nicefrac[]{1}{2}})$ is incorrect \cite{naor}.
We rectify this and note that a correct proof requires a different argument than the one appearing in \cite{khuller1999budgeted}.
This is summarized in the following theorem.
\begin{theorem}\label{thrm:CorrectModifiedGreedy}
The Modified Greedy algorithm of Kuller {\em et. al.} \cite{khuller1999budgeted} for the \SK problem achieves an approximation of $(1-e^{-\nicefrac[]{1}{2}})$.
\end{theorem}
Building upon the correct proof of Theorem \ref{thrm:CorrectModifiedGreedy}, we present a remarkably simple algorithm that chooses the best between the greedy algorithm and the best pair of elements.
Our algorithm retains the same running time of $O(n^2)$ as the greedy algorithm and the Modified Greedy algorithm of \cite{khuller1999budgeted}, but achieves an improved approximation guarantee of $\approx 0.453647$ (whereas the Modified Greedy algorithm provides a guarantee of only $1-e^{-\nicefrac{1}{2}}\approx 0.393$).
In fact, the number of oracle queries used by our algorithm is $3n^2/2+n$.
This is summarized in the following theorem.
\begin{theorem}\label{thrm:ModifiedSquared}
The \SK problem admits an algorithm that runs in time $O(n^2)$ and achieves an approximation of $0.453647$.
\end{theorem}
%While proving Theorem \ref{thrm:ModifiedSquared}, we discovered that the proof that the fast algorithm of \cite{khuller1999budgeted} achieves an approximation guarantee of $(1-e^{-\nicefrac[]{1}{2}})$ is incorrect \cite{naor}.
%We also rectify this and present a different argument that proves the claimed guarantee of \cite{khuller1999budgeted}.

\noindent {\bf{Amplification:}} We present a general method for ``amplifying'' algorithms for \SK.
Specifically, given any black box algorithm $\mathcal{A}$ that obtains an approximation guarantee of $r$ and a number $k\in \mathbb{N}$ of executions, we show how to obtain a new algorithm for \SK with a better approximation than $r$ and a running time that equals $k$ times the running time of the black box algorithm plus an additional  $3n^2/2+n$ oracle queries.
This is summarized in the following theorem.
\begin{theorem}
	\label{thrm:Amplification}
Let $\mathcal{A}$ be an algorithm for the \SK problem achieving an approximation of $r$, where $r< \nicefrac[]{1}{2}$.
Let $k\geq \max \left\{2, -\frac{\log_2(A(\ln 2))}{\log_2(\nicefrac[]{1}{r} -1 ) }+1 \right\}$.
Then there exists an algorithm that uses $3n^2/2+n$ oracle queries plus $k$ times the running time of $\mathcal{A}$ and achieves an approximation of $r^*$, where:
\begin{enumerate}
\item $r^* = \max _{0< \alpha \leq \ln{2}} \left\{ \min \left\{ 1-e^{-\alpha},D(\alpha)+(1-r)\left( \frac{2+\nu(\alpha)}{1+\nu(\alpha)} (1-D(\alpha)) -1\right)\right\}\right\}$.
\item $ D(\alpha)=(1-e^{-\alpha})/B(\alpha)$.
\item $\nu(\alpha ) = 2^{-\frac{\log _2 {A(\alpha)}}{k-1}}-1$.
\item $ A(\alpha) = \frac{1}{1-e^{-\alpha}}-\frac{1}{B(\alpha)}$.
\item $ B(\alpha)=1-e^{-\frac{1}{\alpha}}$.
\end{enumerate}
\end{theorem}

Given any black box algorithm $\mathcal{A}$ for \SK Theorem \ref{thrm:Amplification} can be used to answer two interesting questions: $(1)$ given an upper bound on the running time of the algorithm what is the best approximation that can be obtained using our amplification framework? $(2)$ given a target approximation guarantee what is the required running time using our amplification framework?
For example, choosing the black box algorithm $\mathcal{A}$ to be the algorithm whose existence is guaranteed by Theorem \ref{thrm:ModifiedSquared}, amplifying it with $k=6$ we can obtain an algorithm achieving an improved approximation of $0.48$ and only uses  $10.5n^2+7n$ oracle queries.
Moreover, one can repeatedly apply the amplification, each time with a suitable choice of $k$, obtaining an approximation guarantee that approaches $\nicefrac[]{1}{2}$.

%Theorem \ref{thrm:Amplification} allows us to amplify any algorithm, in particular the algorithm whose existence is guaranteed in Theorem \ref{thrm:ModifiedSquared}.
%Two interesting conclusion, for example, that can be derived from Theorems \ref{thrm:Amplification} and \ref{thrm:ModifiedSquared} are: $(1)$ applying the amplification with $k=6$ to the algorithm whose existence is guaranteed in Theorem \ref{thrm:ModifiedSquared} results in an improved approximation of $0.48$ and a running time of
%
%Theorems \ref{thrm:ModifiedSquared} and \ref{thrm:Amplification} enable us to derive fast algorithms for \SK, by applying the ``amplification'' of Theorem \ref{thrm:Amplification} repeatedly several times.
%For example, one can achieve an approximation of $(1-e^{-\nicefrac[]{2}{3}})\approx 0.4866$ in time {\color{red}{???}}.
%
%%We present a fast and simple algorithm for \SK which retains the fast running time of the greedy algorithm, {\em i.e.}, $O(n^2)$, and achieves an improved approximation guarantee of $(1-e^{-\nicefrac[]{2}{3}})\approx 0.4866$, improving upon the fast algorithm of Khuller {\em et. al.} \cite{khuller1999budgeted}.
%%%We aim to find a {\em fast} and {\em simple} algorithm for the problem of maximizing a monotone submodular function given a knapsack constraint.
%%%We present an algorithm whose running time is that of the greedy algorithm, {\em i.e.}, $O(n^2)$, matching the running time of the fast algorithm of \cite{khuller1999budgeted}, and achieving an improved approximation of $(1-e^{-\nicefrac[]{2}{3}})\approx 0.487$.
%%%This is summarized in the following theorem.
%%\begin{theorem}\label{thrm:OurAlgorithm}
%%There exists an algorithm for the \SK problem achieving an approximation of $(1-e^{-\nicefrac[]{2}{3}})$ whose running time is $O(n ^2)$.
%%\end{theorem}
%%Along the way, we discovered that the proof of the fast algorithm of \cite{khuller1999budgeted} is incorrect \cite{naor}.
%%We rectify this and present a different argument that proves the claimed guarantee of \cite{khuller1999budgeted}.

\paragraph*{Our Techniques}
We adopt a purely combinatorial approach, thus deviating from the recent line of work of \cite{badanidiyuru2014fast,Alina2017}.
Both our results, an improved fast and simple algorithm as well as the amplification, are inspired by the tight (but slow) algorithm of \cite{khuller1999budgeted,sviridenko2004note} for \SK.
We require two new insights that allow us to obtain our results.

The first insight relates to how algorithms for \SK can be analyzed.
At the heart of the analysis of the tight (but slow) algorithm of \cite{khuller1999budgeted,sviridenko2004note} lies the observation that as long as no element of the optimal solution was dropped by the greedy ``bang per buck'' algorithm\footnote{The greedy ``bang per buck'' is formally described in Section \ref{sec:Preliminaries}}, value is accumulated at a rate that depends on the fraction of the budget $\beta$ that was used so far.
Formally, if the greedy ``bang per buck'' uses a fraction $\alpha$ of the budget and it did not drop any element that belongs to the optimal solution so far, then its value is at least a fraction of $(1-e^{-\alpha})$ of the value of the optimal solution.
Once the first element of the optimal solution is dropped, no further analysis of the algorithm is given in \cite{khuller1999budgeted,sviridenko2004note}.

We present an improved approach to analyzing the greedy ``bang per buck'' algorithm and show that in certain cases value can still be accumulated even after elements of the optimal solution were dropped.
This plays a crucial role in all of our results: $(1)$ a correct proof to the fast algorithm of \cite{khuller1999budgeted} (Theorem \ref{thrm:CorrectModifiedGreedy}); $(2)$ an improved fast and simple algorithm (Theorem \ref{thrm:ModifiedSquared}); and $(3)$ our amplification framework (Theorem \ref{thrm:Amplification}).

The second insight relates to the design of the amplification framework.
The tight (but slow) algorithm of \cite{khuller1999budgeted,sviridenko2004note} needs to find a small (of size at most three) subset of the most valuable elements in the optimal solution, and extend it by the greedy ``bang per buck'' algorithm.
This is done by simple enumeration over all small subsets, thus resulting in a tight (but slow) algorithm that runs in $ O(n^5)$ time.
The crucial point in the analysis is that the algorithm needs to find the {\em exact} subset.
Hence, since the optimal solution is not known, enumeration is used over the $O(n^3)$ such subsets.

We, on the other hand, adopt a different approach where it is enough to find a small subset that is {\em comparable} in both value and cost to the most valuable small subset in the optimal solution.
We show that there is only a {\em constant} number of such comparable small subsets.
This allows us, when combined with the first insight, to devise our method for amplifying any algorithm for the \SK problem: improving its approximation factor
by running it a {\em constant} (the parameter $k$ from Theorem \ref{thrm:Amplification}) number of times on meticulously selected instances.
%with only a small loss in the running time.

\paragraph*{Related Work}

A simple greedy algorithm which yields an $(1-e^{-1})$ approximation for the special
case of \SK with uniform costs (i.e. cardinality constraint), along with a matching 
lower bound in the oracle model is known since the late 70's due to the
works of Nemhauser, Wolsey and Fisher \cite{Nemhauser1978}\cite{NW78}.
Their works also achieved a $\nicefrac[]{1}{2}$ approximation for the problem
of maximizing a monotone submodular function subject to a {\em matroid}
constraint \cite{FNW78}. 
An optimal $(1-e^{-1})$ approximation for the latter problem was only attained in the 2000's with 
the introduction of the {continuous greedy} and matching 
	{\em pipage rounding } \cite{CCPV11}.

The problem can also be generalized to the problem of maximizing submodular function
subject to {\em multiple} knapsack constraints for which Kulik et. al. provided a $(1-e^{-1}-\epsilon)$ approximation for every $\epsilon>0$ \cite{KST13}.  


\paragraph*{Paper Organization}
Section \ref{sec:Preliminaries} contains needed notations and description of previous algorithms.
Section \ref{sec:ModifiedGreedy} contains a correct proof of the fast algorithm of Khuller {\em et. al.} \cite{khuller1999budgeted}, whereas Section \ref{sec:Modified2Greedy} contains our improved fast algorithm.
Lastly, Section \ref{sec:Amplification} describes our amplification method. 

\section{Preliminaries}\label{sec:Preliminaries}
First, we start with some notations that will enable us to simplify the presentation of the algorithms and their proofs.
We use the notation of $f(B|A)$ to denote the marginal value of $B$ with respect to $A$, {\em i.e.}, $f(A\cup B)-f(A)$.
When it is clear from the context we use $x$, where $x\in U$, to denote the set $\{ x\}$.
For any subset $A\subseteq U$ we denote by $c(A)\triangleq \sum _{x\in A}c(x)$.
Moreover, for an ordered set $A = \{a_1, \dots, a_k\}$ we use $A_i$ to denote its prefix of size $i$, {\em i.e.}, $\{a_1, \dots, a_i\}$ ($A_0$ is set to the empty set).
Since $f$ is monotone and non-negative, we can assume without loss of generality that $f(\emptyset)=0$.

Second, let us present known fast algorithms for \SK.
The greedy algorithm for \SK which uses the ``bang per buck'' greedy rule appears in Algorithm \ref{alg:greedy}.
It receives as parameters the ground set $U$, the submodular objective $f$ (via its value oracle), the cost function $c$, and the budget $\beta$.

\begin{algorithm}[H]
\caption{Greedy$(U, f, c, \beta)$}
\label{alg:greedy}
$S \leftarrow \emptyset$
\\
\While{$U \neq \emptyset$}{
	$e' \leftarrow \displaystyle{\argmax}\left\{\nicefrac[]{f(e|S)}{c(e)}:e\in U\right\}$
	\\
	$U \leftarrow U \setminus \{e'\}$
	\\
	\If{$c(S \cup \{e'\})\leq \beta$}{
		\label{line:dropped}
		$S \leftarrow S \cup \{e'\}$
	}
}
\Return{S}.
\end{algorithm}

If the condition in line \ref{line:dropped} is evaluated to false we say that
the algorithm \emph{dropped} the element $e'$.
It is well known that in general the approximation ratio of Algorithm \ref{alg:greedy}
can be arbitrarily bad.

Khuller {\em et. al.} \cite{khuller1999budgeted} presented the Modified Greedy algorithm, which returns the best between the greedy algorithm (Algorithm \ref{alg:greedy}) and the single best element.
This appears in Algorithm \ref{alg:mgreedy} (its input is identical to that of Algorithm \ref{alg:greedy}).

\begin{algorithm}[H]
\caption{Modified Greedy$(U, f, c, \beta)$}
\label{alg:mgreedy}

$S \leftarrow \text{Greedy}(U, f, c, \beta)$
\\
$T \leftarrow \argmax\left\{S, \argmax \left\{f(e):e\in U, c(e)\leq \beta \right\} \right\}$
\\
\Return{$T$}
\end{algorithm}

Third and last, we present a lemma (can be derived from, {\em e.g.}, \cite{khuller1999budgeted}) that enables one to analyze the performance of the greedy algorithm (Algorithm \ref{alg:greedy}).
\begin{lemma}
\label{lemma:sub-main}
Let $A = \{a_1, \dots, a_k\}$ and $B$ be two subsets such that for all $1 \leq i \leq k$
and for all $e \in B$ it holds that
$\frac{f(a_i|A_{i-1})}{c(a_i)} \geq \frac{f(e|A_{i-1})}{c(e)}$.
Then, $f(A) \geq (1 - e^{-\frac{c(A)}{c(B)}})f(B)$.
\end{lemma}

The proof for the lemma is given in Appendix  \ref{appendix:omitted}. We note that 
the proof only uses standard techniques from previous works.

A particular corollary of the above lemma is that if $A$ is the set of elements chosen by the greedy algorithm in
the $k$ first iterations and $B$ is any subset such that no element from $B$ was dropped by the greedy algorithm in the first $k$ iterations,
then all the condition of Lemma~\ref{lemma:sub-main} are met and it can be applied to lower bound the value of $A$.







\section{Correct Proof for Approximation Factor of Modified Greedy}\label{sec:ModifiedGreedy}
We note that the proof given by Khuller {\em et. al.} \cite{khuller1999budgeted} that the Modified Greedy algorithm (Algorithm~\ref{alg:mgreedy}) provides an approximation of $ (1-e^{-\nicefrac[]{1}{2}})$ for \SK is flawed.
{{The flaw originates from applying Lemma~\ref{lemma:sub-main} incorrectly for a set for which the greedy algorithm (Algorithm~\ref{alg:greedy}) might have dropped elements.
See the last statement in the proof of Theorem 3 in~\cite{khuller1999budgeted} (page 42, right column, line 10 from the bottom).}}
To give a correct proof we incorporate a new insight to
previous techniques for analyzing of the greedy algorithm
(e.g \cite{khuller1999budgeted} and \cite{LB10}). While previous approaches only
showed lower
bounds on the value of elements selected by the greedy before an
 element from the optimal solution has been dropped, we
further look into the value attained by the algorithm after
an element from the optimal solution has been dropped.
%We note that in order to give a correct proof for the performance of the algorithm a somewhat different argument is required.

Let $O$ be an optimal solution, and recall that $T$ is the output of Algorithm~\ref{alg:mgreedy}.

%We analyze the Modified Greedy algorithm (Algorithm~\ref{alg:mgreedy}) of \cite{khuller1999budgeted}.
%Our analysis matches the $(1 - e^{-1/2})$-approximation claimed in \cite[]{khuller1999budgeted}, correcting the logical flaw in the analysis.

%Let $O$ be an optimal set, and recall that $T$ is the output of the Modified Greedy algorithm.
%The following theorem proves that Algorithm~\ref{alg:mgreedy} is a $(1-e^{-1/2})$-approximation.

%\begin{theorem}
%	\label{theorem:mgreedy}
%	The Modified Greedy algorithm (Algorithm~\ref{alg:mgreedy}) achieves an approximation of $(1 - e^{-1/2})$ for the \SK problem.
%\end{theorem}

\begin{proof}[Proof of Theorem \ref{thrm:CorrectModifiedGreedy}]
	We can assume that $f(e) \leq (1 - e^{-1/2})f(O)$ for every $e \in U$, otherwise the value of $e$, and hence of $T$, is large enough and the theorem holds.
	Recall that by the definition of Algorithm~\ref{alg:mgreedy} $S$ is the output of the greedy algorithm, and let $e^*$ be the most valuable element.
	
	Consider two cases depending on how many elements of $O$ are dropped by the greedy algorithm, i.e., $|O \setminus S|$.
	If $|O \setminus S| \leq 1$, then
	$$
	f(O) \leq f(O \cup S) \leq f(O \setminus S) + f(S) \leq f(e^*) + f(S).
	$$
	The above implies that either $f(S) \geq \frac{1}{2}f(O)$ or $f(e^*) \geq \frac{1}{2}f(O)$ concluding the proof for this case.
	Otherwise, $|O \setminus S| \geq 2$.
	Denote by $a$ and $b$ the first and second elements of $O$ the greedy algorithm drops, {\em i.e.} $a$ is the considered element at the first time the condition in line~\ref{line:dropped} of Algorithm~\ref{alg:greedy} is evaluated to false and $b$ is the considered element the second time the same condition is evaluated to false.
	Denote by $A$ the set of elements chosen by the algorithm just before dropping $a$ and by
	$B$ the set of elements chosen right after dropping $a$ and before dropping $b$.
	If $f(A) \geq (1 - e^{-1/2})f(O)$ then the theorem holds.
	Otherwise, $f(A) = (1 - e^{-(1/2 - \delta)})f(O)$ where $0 < \delta \leq \nicefrac[]{1}{2}$.

	To prove the theorem we need to infer from the above bounds on both size and value.
	First, we focus on bounds on the size:
	\begin{align}
		\label{mgreedy:ineq1}
		c(A) \leq (\nicefrac[]{1}{2} - \delta)c(O)
		\\
		\label{mgreedy:ineq2}
		c(a) > (\nicefrac[]{1}{2} + \delta)c(O)
		\\
		\label{mgreedy:ineq3}
		c(b) \leq c(O \setminus a) \leq (\nicefrac[]{1}{2} - \delta)c(O)
	\end{align}
	Inequality \eqref{mgreedy:ineq1} upper bounds the size of $A$ and follows from  Lemma~\ref{lemma:sub-main} since if it does not hold then $f(A) > (1 - e^{-(1/2 - \delta)})f(O)$.
%Specifically, if $c(A) > (\nicefrac[]{1}{2} - \delta)c(O) $ one can apply Lemma~\ref{lemma:sub-main} on $A$ and $O$.
%Note that all conditions of Lemma~\ref{lemma:sub-main} are satisfied by the definition of $A$ and thus it can be applied in this case.
	Inequality \eqref{mgreedy:ineq2} lower bound the size of $a$ and follows from the fact that $a$ was dropped by the greedy algorithm, {\em i.e.}, $c(A \cup \{a\}) > \beta \geq c(O)$. Inequality \eqref{mgreedy:ineq3} is due to inequality \eqref{mgreedy:ineq2}, since $c(O \setminus \{a\}) = c(O) - c(a)$ and $\{b\} \subseteq O \setminus \{a\}$.
	The above inequalities imply a lower bound  on the size of $B$:
	\begin{equation}
		\label{mgreedy:ineq4}
		c(B) > (2\delta)c(O).
	\end{equation}

	Second, we focus on bounds on the value.
	We need to lower bound the value of $O$ after $a$ is dropped with respect to $A$, {\em i.e.}, $f(O \setminus a | A)$:
	\begin{equation}
		f(O \setminus a | A) \geq f(O) - f(a) - f(A) \geq (e^{-1/2} - 1 + e^{-(1/2 - \delta)})f(O),
	\end{equation}
	where the inequality follows from the submodularity and monotonicity of $f$.
	Furthermore, we need to lower bound the value of the elements the greedy algorithm chooses after discarding $a$ and up to discarding $b$ with respect to $A$, {\em i.e.},	
	\begin{align}
		\label{ineq:B}
		f(B|A)	& \geq (1 - e^{-\frac{2\delta}{1/2 - \delta}})f(O \setminus a | A)
		\\ 		& \geq (1 - e^{-\frac{2\delta}{1/2 - \delta}})(e^{-1/2} - 1 + e^{-(1/2 - \delta)})f(O).
		\nonumber
	\end{align}	
	Inequality~(\ref{ineq:B}) follows from applying Lemma~\ref{lemma:sub-main} on the sets $B$ and $O \setminus a$, and considering the submodular function $f(X|A)$ for every $X \subseteq U$.
	
	Finally, we lower bound the value of the output:
	\begin{equation}
		f(T) \geq f(A \cup B) = f(A) + f(B | A).
	\end{equation}	
	Substituting $f(A)$ with $(1 - e^{-(1/2 - \delta)})f(O)$ and lower bounding $f(B|A)$ using~(\ref{ineq:B}) results in the following:
	\begin{equation}
		\label{ineq:T}
		f(T) \geq (1 - e^{-(1/2 - \delta)})f(O) + (1 - e^{-\frac{2\delta}{1/2 - \delta}})(e^{-1/2} - 1 + e^{-(1/2 - \delta)})f(O).
	\end{equation}
	One can verify that the expression in~(\ref{ineq:T}) is at least $(1-e^{-1/2})f(O)$ for $0 < \delta \leq \nicefrac[]{1}{2}$.
\end{proof}



\section{The Modified\textsuperscript{2} Greedy Algorithm}\label{sec:Modified2Greedy}
We present a remarkably simple adaptation of the Modified Greedy algorithm of \cite{khuller1999budgeted} (Algorithm \ref{alg:mgreedy}) that retains the same running time of $O(n^2)$ for which we can prove a better approximation guarantee of $0.453647$.
Our algorithm returns the best between the greedy algorithm (Algorithm \ref{alg:greedy}) and highest value pair of elements.
We denote this algorithm by Modified\textsuperscript{2} Greedy and it appears as Algorithm \ref{alg:mmgreedy}.
We prove that Algorithm \ref{alg:mmgreedy} achieves the guarantee claimed in Theorem \ref{thrm:ModifiedSquared}.

%The modified greedy algorithm returns the best solution between the greedy solution and the best singleton.
%Since scanning all pairs of elements rather than just singletons does not affect the running time of the modified greedy algorithm a natural question to ask is whether considering pairs of elements result in strictly better algorithm.
%Here we do not give a complete answer to this question but we are able to improve our analysis for this algorithm.

%The \emph{Modified$^2$ Greedy} algorithm~\ref{alg:mmgreedy}
%is similar to the  modified greedy algorithm,
%except it considers the best pair of element rather than the best singleton element.


\begin{algorithm}
	$S \leftarrow \text{greedy}(U, f, c, \beta)$
	\\
	$T \leftarrow \arg\max\{S, \argmax \{ f(\{ e_1,e_2\}):e_1,e_2\in U, c(\{e_1, e_2\} ) \leq \beta \} \}$
	\\
	\Return{T}
	%
	\caption{Modified\textsuperscript{2} Greedy$(U, f, c, \beta)$}
	\label{alg:mmgreedy}
\end{algorithm}

\begin{proof}[Proof of Theorem \ref{thrm:ModifiedSquared}]
Let $0<\varepsilon \leq \nicefrac[]{1}{6}$ be an absolute constant to be determined later, and let $O$ be an optimal solution to the problem.
If there is a pair of elements $e_1, e_2 \in U$ whose value is high enough, {\em i.e.}, $f(\{e_1, e_2\}) \geq (1 - e^{-(\nicefrac[]{1}{2} + \varepsilon)})f(O)$, then the theorem follows.
Thus, let us assume that $f(\{e_1, e_2\}) < (1 - e^{-(\nicefrac[]{1}{2} + \varepsilon)})f(O)$ for every pair of elements $e_1, e_2 \in U$.

Similarly to the proof of Theorem~\ref{thrm:CorrectModifiedGreedy}, we can assume that the greedy algorithm discards at least three elements from $O$, {\em i.e.}, $|S \setminus O| \geq 3$.
Otherwise, we are guaranteed that $f(T) \geq \nicefrac[]{1}{2}\cdot f(O)$.
Let $a$ be the first element in $O$ that was dropped by the algorithm, {\em i.e.}, the first time the condition in line~\ref{line:dropped} was false, and let $A$ be the set of elements chosen by the algorithm just before dropping $a$.
If $f(A) \geq (1 - e^{-(\nicefrac[]{1}{2} + \varepsilon)})f(O)$ the theorem holds.
Otherwise, denote $f(A) = (1 - e^{-(1/2 + \varepsilon - \delta)})f(O)$ where $0 < \delta \leq \nicefrac[]{1}{2} + \varepsilon$.

Lemma~\ref{lemma:sub-main} implies that:
\begin{equation}
	\label{mmgreedy:ineq1}
	c(A) \leq (\nicefrac[]{1}{2} + \varepsilon - \delta)c(O),
\end{equation}
otherwise $f(A) > (1-e^{-(1/2 + \varepsilon - \delta)})f(O)$.
Moreover, since $a$ was dropped we confirm that:
\begin{equation}
	\label{mmgreedy:ineq2}
	c(a) > (\nicefrac[]{1}{2} -\epsilon + \delta)c(O),
\end{equation}
since $c(A \cup \{a\}) > \beta \geq c(O)$.
We say that an element $e$ is \emph{big} if $c(e) \geq (\nicefrac[]{1}{4} + \nicefrac[]{\varepsilon}{2} - \nicefrac[]{\delta}{2})c(O)$,
otherwise it is \emph{small}.
Note that $a$ is big since $\varepsilon \leq \nicefrac[]{1}{6}$ and $\delta > 0$.

Note that $O\setminus\{a\}$ contains at most one big element (if that is not the case then the size of two big elements is at least $(\nicefrac[]{1}{2} + \varepsilon - \delta)c(O)$), in contradiction to (\ref{mmgreedy:ineq2}).
Thus, since the greedy algorithm dropped at least three elements from $O$, it must be the case that at least one of them is small.
Let $b$ be the first such element, and let $B$ be the set of elements chosen by the
algorithm right after dropping $a$ and just before dropping $b$.
Also, denote by $C$ the subset of small elements in $O$,
{\em i.e.}, $C = \{e \in O : c(e) < (\nicefrac[]{1}{4} + \nicefrac[]{\varepsilon}{2} - \nicefrac[]{\delta}{2})c(O)\}$.
We note that:
\begin{equation}
\label{mmgreedy:ineq3}
c(C) \leq (\nicefrac[]{1}{2} + \varepsilon - \delta)c(O),
\end{equation}
since $a$ is big.
From the above inequalities we can derive a lower bound on the size of $B$:
\begin{equation}
	\label{mmgreedy:lower-bound-cB}
	c(B) \geq (\nicefrac[]{1}{4} - \nicefrac[]{(3\varepsilon)}{2} + \nicefrac[]{(3\delta)}{2})c(O).
\end{equation}

First, we wish to lower bound $f(C)$.
Since $O$ contains at most two big elements, and the fact that the value of any pair of elements cannot exceed $(1 - e^{-(\nicefrac[]{1}{2} + \varepsilon)})f(O)$, we can conclude that $f(C) \geq e^{-(\nicefrac[]{1}{2} + \varepsilon)}$ (follows from the submodularity of $f$).

Second, we aim to lower bound the marginal value of $B$ with respect to $A$.
To that end we apply Lemma~\ref{lemma:sub-main} on $B$ and $C$ and choose the submodular function to be $f(X|A)$ for every $X \subseteq U$.
This gives us the following:
\begin{equation}
	\label{mmgreedy:lower-bound-B-given-A}
	f(B|A) \geq
	(1-e^{-\frac{1-6\varepsilon+6\delta}{2+4\varepsilon-4\delta}})
	\left[
	e^{-(\nicefrac[]{1}{2} + \varepsilon)}
	- (1 - e^{-(\nicefrac[]{1}{2} + \varepsilon - \delta)})
	\right]f(O)
\end{equation}

Finally, we would like to lower bound $f(T)$.
Note that if $f(A) \geq (1 - e^{-(\nicefrac[]{1}{2} + \varepsilon)})f(O)$ we can lower bound $f(T)$ by $(1 - e^{-(\nicefrac[]{1}{2} + \varepsilon)})f(O)$.
Otherwise,
\begin{align}
	&f(A \cup B)
	 = \nonumber
	f(A) + f(B|A)
	\\ & \geq
	(1-e^{-(\nicefrac[]{1}{2} + \varepsilon - \delta)})f(O)
	+
	(1-e^{-\frac{1-6\varepsilon+6\delta}{2+4\varepsilon-4\delta}})
	\left[
	e^{-(1/2 + \varepsilon)}
	- (1 - e^{-(1/2 + \varepsilon - \delta)})
	\right] f(O).
\end{align}
Thus, for a given fixed $\varepsilon > 0$ we have:
\begin{equation}
	\frac{f(T)}{f(O)} \geq  \min \begin{cases}
		(1 - e^{-(1/2 + \varepsilon)})
		\\
			\displaystyle{\min_{\delta}}\left\{
			(1-e^{-(\nicefrac[]{1}{2} + \varepsilon - \delta)})
			+
			(1-e^{-\frac{1-6\varepsilon+6\delta}{2+4\varepsilon-4\delta}})
			\left[
			e^{-(\nicefrac[]{1}{2} + \varepsilon)}
			- (1 - e^{-(\nicefrac[]{1}{2} + \varepsilon - \delta)})
			\right]\right\}
	\end{cases}
\end{equation}
%\begin{equation}
%	f(T) \geq \min \begin{cases}
%		(1 - e^{-(1/2 + \varepsilon)})f(O)
%		\\
%			\displaystyle{\min_{\delta}}
%			(1-e^{-(\nicefrac[]{1}{2} + \varepsilon - \delta)})f(O)
%			+
%			(1-e^{-\frac{1-6\varepsilon+6\delta}{2+4\varepsilon-4\delta}})
%			\left[
%			e^{-(\nicefrac[]{1}{2} + \varepsilon)}
%			- (1 - e^{-(\nicefrac[]{1}{2} + \varepsilon - \delta)})
%			\right]f(O),
%	\end{cases}
%\end{equation}
for $0 < \delta \leq \nicefrac[]{1}{2} + \varepsilon$.
Optimizing over $0 < \epsilon \leq \nicefrac[]{1}{6}$ we get that $\varepsilon = 0.10449$ and $\delta = 0.388031$ resulting in an approximation of $0.453647$.
\end{proof}


\section{Amplification Algorithm}\label{sec:Amplification}
\def\pLarge{P_{\text{large}}}
\def\pVal{P_{\text{val}}}
\def\MGreedy{Modified$^2$Greedy}
\def\BOTAlg{BestOfThree}
\def\mA{\mathcal{A}}

In this section we show a simple algorithm, which given
an $r$-approximation $\mA$ for \SK, $r<1/2$ can
be used to derive an $r'$-approximation, $r<r' <1/2$, for
\SK using  a constant number of calls
for $\mA$ and $\frac{3n^2}{2}+n$ oracle queries.

Algorithm \ref{algorithm:amplify} receives
an approximation algorithm $\mA$ for $\SK$,for
a universe $U$, a monotone non-negative
submodular function $f:2^U \rightarrow \mathbb{R}$,
a non-negative cost function $c:U \rightarrow \mathbb{R}$, and a budget $\beta$.
The algorithm also gets two scalar parameters $\epsilon$
and $\rho$.  These two parameters control the approximation
ratio and complexity of the algorithm, and should be set according
to the claims in this section to obtain the required approximation
ratio and complexity.

Broadly speaking,  the algorithm finds the pair of  elements with maximal value $\pVal$. It then divides all pairs of elements $\{a,b\} \subseteq U$ for which
$\frac{f(\{a,b\})}{f(\pVal)} \geq \rho$ into {\em buckets}, where the subsets
in each bucket have the same value to a factor of $(1+\epsilon)$.
The algorithm then extends the set of smallest cost in  each bucket
to a solution using the algorithm $\mA$.  The algorithm returns the
best between the solutions found using the buckets and $\mA$, the pair of
elements with maximal profit, the single element with maximal profit, and the result of the greedy algorithm over the input instance.

We remark that while the proofs of our claims regarding the amplification framework are a bit long and involved, the resulting amplified algorithm is simple and comes down to choosing a constant number of pairs of elements and extending them to complete solutions using the given (unamplified) algorithm as a black-box.

\begin{algorithm}
	\caption{Amplify($\mA, U, f, c, \beta, \epsilon, \rho$)}
	\label{algorithm:amplify}
	% Initialization
	\tcp{Initialization}
	$\pVal \leftarrow \argmax_{ \{a,b\} \subseteq U, c(\{a,b\})\leq \beta } f(\{a,b\})$
	\\ $w_{\max}\leftarrow f(\pVal)$
	\\ $i_{\max} \leftarrow \floor{\log_{1 + \epsilon} \frac{1}{\rho}}$
	\\
	\tcp{Buckets}
	\For{$i \in \{0,\dots,i_{\max}\}$}{
		$B_i = \left\{ \{a,b\}\subseteq U| c(\{a,b\})\leq \beta, \rho(1+\epsilon)^i \leq \frac{f(\{a,b\})}{w_{\max}}  < \rho (1+\epsilon)^{i+1} \right\}$
		\\
		$P_i \leftarrow \argmin_{\{a,b\}\in B_i} c(\{a,b\})$
%	}
%	\tcp{Candidate Solutions}
%	\For{$i \in \{0,\dots,i_{\max}\}$}{
\\
		$S_i \leftarrow P_i \cup \mA(U , f_{P_i}, c, \beta - c(P_i))$
	}
	$G \leftarrow \text{Greedy}(U, f, c, \beta)$ \label{amplify:greedy}
	\\
	$T \leftarrow \argmax_{\{a\}\subseteq U| c(a)\leq \beta} f(\{a\})$ \label{amplify:singletons}
	\\
	\Return $\argmax_{S \in \{S_0, S_1, \ldots, S_{i_{\max}}, G, \pVal,T\}} f(S)$
	%

\end{algorithm}

The following function, which already appeared on Theorem \ref{thrm:Amplification}, will be useful throughout the analysis of the algorithm:

$$ B(\alpha)=1-e^{-\frac{1}{\alpha}}$$
$$ A(\alpha) = \frac{1}{1-e^{-\alpha}}-\frac{1}{B(\alpha)}$$
 $$D(\alpha)=(1-e^{-\alpha})/B(\alpha)$$

 \begin{lemma}
 	\label{lemma:amplification}
 	Given an $r$-approximation $\mA$ for $\SK$ where $r<\frac{1}{2}$.
 	For any $0< \alpha \leq  \ln 2$, input $U,f,c,\beta$ of $\SK$,
 	and $\epsilon$ such that $0<\epsilon\leq \frac{1-2r}{r}$, then executing
 	Algorithm \ref{algorithm:amplify}  with parameters
 	$(\mA, U, f, c, \beta, \epsilon, A(\alpha))$ returns
 	$$min\left(1-e^{-\alpha}, D(\alpha) + (1-r) \left( \frac{2+\epsilon}{1+\epsilon}(1- D(\alpha)) -1 \right) \right)$$
 	approximation for the input instance.
 \end{lemma}



\begin{proof}

	Fix an optimal solution $O$.
	 If $|O|\leq 2$ the algorithm returns an optimal solution and the Lemma holds.
	Therefore we can assume $|O| \geq 3$.  Denote by $\pLarge$ the set of  two  elements in $O$ with highest cost.
	
	Let $L\beta$ be the highest cost of element in $O$.  If $L\leq 1-\alpha$ then
	by Lemma \ref{lemma:sub-main} and the monotonicity of $f$: $f(G) \geq (1-e^{-\alpha}) f(O)$  (recall that $G$ is the result of
	greedy in line \ref{amplify:greedy} of the algorithm and thus Lemma \ref{lemma:sub-main} can be applied on the appropriate prefix of $G$). Therefore, the algorithm
	returns a solution of value at least $(1-e^{-\alpha})f(O)$ and the Lemma holds. Therefore we can assume  $L> 1-\alpha$.
%
%	{\color{red}
%	consider $G'\subseteq G$ to be the subset selected by the greedy (in line \ref{amplify:greedy})
%	just before the first element from $O$ has been dropped ($G' = G$ if no such element exists).
%	Due to the greedy procedure selection of elements the conditions of Lemma \ref{lemma:sub-main}
%	hold for $G'$ (as $A$)  and $O$ (as $B$), therefore,
%}  $f(G) \geq (1-e^{-\alpha}) f(O)$. Therefore, the algorithm
%	returns a solution of value at least $(1-e^{-\alpha})f(O)$ and the Lemma holds. Hence, we can assume  $L> 1-\alpha$.
	
	If $O\setminus \pLarge \subseteq G$, then
	$f(G)+ f(\pVal)\geq f(O\setminus \pLarge) + f(\pLarge) \geq f(O)$.
	Therefore $f(G)\geq \frac{f(O)}{2}$ or $f(\pVal)\geq \frac{f(O)}{2}$
	thus the Lemma holds (note that $\alpha \leq \ln 2$ and therefore
	$1-e^{-\alpha} < \frac{1}{2}$). Thus we can assume that
	$O\setminus \pLarge \nsubseteq G$.
	
	The following corollary states that if $f(\pLarge)$ is fairly small
	with respect to $f(O)$ (equivalently, $f(O|\pLarge)$ is high) then the solution $G$ from the greedy algorithm provides
	the required approximation ratio.
	
	\begin{corollary}
		\label{corollary:cor1}
	If $f(O|\pLarge) \geq D(\alpha) f(O)$ then $f(G)\geq (1-e^{-\alpha})f(O)$.
	\end{corollary}

\begin{proof}
	

	Let $\beta M$ be the third highest cost of element in  $O$.
	As the highest cost of  element in $O$ is $\beta L$,
	using a simple argument we get, $M\leq M(L)$ where $M(L)= \min \{\frac{1-L}{2}, L\}$. Since
	$O\setminus \pLarge \nsubseteq G$, there must be an element from $O \setminus \pLarge$ the greedy in line \ref{amplify:greedy} drops. Since all the elements
	in $O\setminus \pLarge$ are of cost at most $\beta M$ the knapsack must
	already have elements of cost $\beta (1-M)$ at the first time an element from $O\setminus \pLarge$ is dropped. Also, $c(O\setminus \pLarge) \leq \beta(1-L -M)$, therefore by Lemma \ref{lemma:sub-main} and the monotonicity of $f$ we get that:
  	$$f(G)\geq \left(1-e^{- \frac{\beta (1-M) }{c(O\setminus \pLarge)}}\right) f(O\setminus \pLarge)
  			\geq \left(1-e^{- \frac{1-M }{1-L-M}}\right) f(O\setminus \pLarge) .$$
  			
  	We note that the term $\frac{1-M}{1-L-M} = \left(  1- \frac{L}{1-M} \right)^{-1}$ is increasing as a function
  	of $M$  in the range $[0, M(L)]$ (note that $1-L-M >0$ in the range).
  	Therefore $\frac{1-M}{1-L-M} \geq  \frac{1 }{1-L}  \geq \frac{1}{1-(1-\alpha)} = \nicefrac[]{1}{\alpha}$. And conclude
  %	The term  $1-e^{- \frac{1-M(L) }{1-L-M(L)}}$ is increasing as a function of $L$
  %	({\bf this is not trivial, I simply drew the graph on Desmos for that}), therefore
  	%	as $L> (1-\alpha)$ we get
  %		$$f(G)\geq  \left(1-e^{- \frac{1-M(L) }{1-L-M(L)}}\right) f(O\setminus \pLarge)
  	%	\geq \ \left(1-e^{- \frac{1-M(1-\alpha) }{1-(1-\alpha)-M(1-\alpha)}}\right) f(O\setminus \pLarge) $$
  		
  	%As $\frac{2}{3} \leq \alpha \leq \ln 2$ we have $M(1-\alpha) = 1-\alpha$. Combining this with the previous inequality we get
  	  $$f(G)
  	\geq \ \left(1-e^{- \frac{1-M }{1-L-M}}\right) f(O\setminus \pLarge) \geq \left( 1- e ^ {- \frac{1}{ \alpha}} \right)f(O\setminus \pLarge)
  	= B(\alpha)f(O\setminus \pLarge) $$
  	
  	Now, recall that $D(\alpha)= \frac{1-e^{-\alpha}}{B(\alpha)}$,
	$f(O|\pLarge) \geq D(\alpha) f(O)$ by the condition of the
	lemma and $f(O|\pLarge) \leq f(O\setminus \pLarge)$ as
	$f$ is submodular. Using these observations and the last lower
	bound on $f(G)$ we obtain
%	\begin{equation*}
%\begin{array}{rcl}
\begin{align*}
f(G) \geq& B(\alpha)f(O\setminus \pLarge)  \geq
B(\alpha) f(O|\pLarge)\\  \geq&
B(\alpha ) D(\alpha ) f(O)
= B(\alpha)\frac{1-e^{-\alpha}}{B(\alpha) }f(O)= (1-e^{-\alpha})f(O)
\end{align*}
%\end{array}
%	\end{equation*}

	\end{proof}

	\begin{corollary}
		\label{corollary:cor2}
		If $f(G)< (1-e^{-\alpha})f(O)$ and $f(\pVal) < (1-e^{-\alpha})f(O)$ then
		$f(\pLarge) \geq A(\alpha) f(\pVal)$
		\end{corollary}
	\begin{proof}
		As $f(G)< (1-e^{-\alpha})$, the condition of corollary \ref{corollary:cor1} does not hold. Therefore
		$f(O|\pLarge)< D(\alpha) f(O)$. Hence,
		$$f(O)= f(\pLarge) + f(O|\pLarge) < f(\pLarge) + D(\alpha) f(O).$$
		By rearranging the terms and using $f(\pVal) < (1-e^{-\alpha})f(O)$
		we get
		$$
		f(\pLarge) > (1-D(\alpha)) f(O) > \frac{1-D(\alpha)}{1-e^{-\alpha}} f(\pVal)
		= \left(\frac{1}{1-e^{-\alpha}} - \frac{D(\alpha)}{1-e^{-\alpha}}\right) f(\pVal) = A(\alpha) f(\pVal).
		$$
		The second inequality uses the observation that $D(\alpha) \leq 1$ for $0\leq \alpha \leq 1$ which can be easily verified. The last equality follows from the definitions
		of $D(\alpha)$ and $A(\alpha)$.
		
			
		\end{proof}

	\begin{corollary}
		\label{corollary:cor3}
			If $f(G)< (1-e^{-\alpha})f(O)$ and $f(\pVal) < (1-e^{-\alpha})f(O)$ then
			there is $i\in \{0, \ldots, i_{\max}\}$ such that
		$$\frac{f(S_i)}{f(O)} \geq D(\alpha) + (1-r)\left( \frac{2+\epsilon}{1+\epsilon} D(\alpha) -1 \right).$$
	\end{corollary}
\begin{proof}
		As the conditions of corollary \ref{corollary:cor2} hold, we have
		$f(\pLarge) \geq A(\alpha) f(\pVal)$.
		 Let $i$ be the minimal integer $i$ such
		that $\frac{f(\pLarge)}{f(\pVal)} < A(\alpha)(1+\epsilon) ^{i+1}$, as $f(\pLarge) \geq A(\alpha) f(\pVal)$ we get $i\geq 0$.
		Recall that $i_{\max}= \floor{\log_{1+\epsilon} \frac{1}{\rho} } =
		\floor{\log_{1+\epsilon} \frac{1}{A(\alpha)}} $ (as the algorithm is used
			with $\rho=A(\alpha)$).
			Therefore,
			$$A(\alpha) (1+\epsilon) ^{i_{\max} +1}
			> A(\alpha) \frac{1}{A(\alpha) } = 1 \geq \frac{f(\pLarge)}{f(\pVal)}$$
			Thus $i\leq i_{\max}$.
			Also $c(\pLarge) \leq c(O) \leq \beta$, and we can conclude that
			$\pLarge \in B_i$ (note that $w_{\max} = f(\pVal)$).
			
			From the definition of $P_i$ we get $c(P_i)\leq c(\pLarge)$ and
			$f(P_i) \geq \frac{1}{1+\epsilon} f(\pLarge)$.
			Let $Q_i  =  \mA(U, f_{P_i}, c, \beta - c(P_i))$.
			As $c(P_i) \leq c(\pLarge)$ we get that $O\setminus \pLarge$
			is a feasible solution for the problem instance given to $\mA$.
			Hence, as $\mA$ is a $r$-approximation we get
			$$f(Q_i | P_i) \geq r f(O\setminus \pLarge| P_i).$$
			Therefore,
			\begin{align*}
			f(S_{i})
			&
			\geq f(P_i) + f(Q_i|P_i)
			\\ &
			\geq f(P_i) + r f(O \setminus \pLarge | P_i)
			\\ &
			\geq f(P_i) + r(f(O) - f(\pLarge) - f(P_i))
			\\ &
			= (1-r)f(P_i) + r(f(O) - f(\pLarge))
			\\ &
			\geq \frac{1-r}{1 + \epsilon}f(\pLarge) + r(f(O) - f(\pLarge))
			\\ &
			= \left(
			\frac{(2 + \epsilon)}{(1 + \epsilon)}(1-r) - 1
			\right)
			f(\pLarge)
			+ rf(O)
			\end{align*}
			
			As $\epsilon\leq \frac{1-2r}{r}$ one can deduce that
			$\left(
			\frac{(2 + \epsilon)}{(1 + \epsilon)}(1-r) - 1
			\right) \geq 0$.
			Also, since $f(G)< (1-e^{-\alpha})f(O)$ by corollary \ref{corollary:cor1} we
			get $f(O|\pLarge)  <D(\alpha)f(O)$, by using $f(O|\pLarge)= f(O) -f(\pLarge)$
			and rearranging  the terms we get
			$f(\pLarge)> (1-D(\alpha)) f(O)$.
			Therefore,
			\begin{align*}
			f(S_i) \geq&
			\left(
			\frac{(2 + \epsilon)}{(1 + \epsilon)}(1-r) - 1
			\right)
			f(\pLarge)
			+ rf(O) \\ \geq &
				\left(
			\frac{(2 + \epsilon)}{(1 + \epsilon)}(1-r) - 1
			\right) (1-D(\alpha ) )f(O) +rf(O)\\
			=& f(O) \left(  D(\alpha) + (1-r)\left(  \frac{(2 + \epsilon)}{(1 + \epsilon)} (1-D(\alpha)) -1\right) \right)
			\end{align*}
			and the corollary immediately follows.
			
		
	\end{proof}
The Lemma follows immediately from corollary \ref{corollary:cor3} as it states
that either $G$, $\pVal$ or $S_i$ for some $i$ would provide the required approximation
ratio.
\end{proof}

\begin{lemma}
	\label{lemma:amp_runtime}
Algorithm \ref{algorithm:amplify} uses $\floor{\log_{1+\epsilon} \frac{1}{\rho}}+1$
	invocation to $\mA$ and up to $\frac{3}{2} n ^2+n$ additional oracle queries.
 \end{lemma}
\begin{proof}
	The number of invocation to $\mA$ immediately follows from the algorithm.
	Beside the invocation to $\mA$ the algorithm runs the greedy procedure
	which uses up to $n^2$ queries. The queries for the initialization and
	buckets phases only considers sets of size $2$, and therefore can be
	implemented by up to $n(n-1)/2\leq n^2/2$ queries. The execution
	of line \ref{amplify:singletons} would take another $n$ queries.
	\end{proof}



\begin{proof}[Proof of Theorem \ref{thrm:Amplification}]
	Let $$\alpha^*  = \argmax _{0<\alpha \leq \ln{2}} \left\{ \min \left\{ 1-e^{-\alpha},D(\alpha)+(1-r)\left( \frac{2+\nu(\alpha)}{1+\nu(\alpha)} (1-D(\alpha)) -1\right)\right\}\right\}.$$
	We obtain the described approximation ratio when running the algorithm
	with $\epsilon = \nu(\alpha^*)$ and $\rho = A(\alpha^*)$.
	By the conditions of the theorem,
	$$k \geq -\frac{\log A(\ln 2)} { \log(\nicefrac[]{1}{r} -1 ) } +1 \geq-\frac{\log A(\alpha*)} { \log(\nicefrac[]{1}{r} -1 ) } +1 .$$
	Where the last inequality uses the fact that $A$ is monotonically decreasing.  Thus,
	$$\epsilon^* = 2 ^{-\frac{\log_2 (A(\alpha^*))}{k-1}} -1 \leq \
	2 ^{-\frac{ \log_2(A(\alpha^*))}{  -\left(\frac{\log_2(A(\alpha* ))}{\log_2(\nicefrac[]{1}{r}-1)} \right) }} -1  =
	2^{\log_2(\nicefrac[]{1}{r}-1)} -1 = \nicefrac[]{1}{r}-2 = \frac{1-2r}{r} .$$
	therefore the conditions of Lemma \ref{lemma:amplification} apply,
	and the approximation ratio follows.
	
	By lemma \ref{lemma:amp_runtime} the number of invocations for $\mA$
	is
	\begin{align*}
	\floor{\log_{1+\epsilon^*} \frac{1}{A(\alpha^*)}}+1 \leq \
	-\log_{1+\epsilon^*} (A(\alpha^*)) +1 =
	\frac{-\log A(\alpha^*)} {\log(1+\epsilon^*)} +1 \\
	= \frac{-\log {A(\alpha^*)}} {\log(2 ^{-\frac{\log_2 (A(\alpha^*))}{k-1}} )} +1
	= \frac{-\log {A(\alpha^*)}} {-\frac{\log_2 (A(\alpha^*))}{k-1}} +1
	= k \end{align*}
	and the number of addition oracle queries is $\nicefrac[]{3n^2}{2} + n $
	as required.
	
\end{proof}


\bibliographystyle{plain}
\bibliography{main}

\appendix
\section{Omitted Proofs}
\label{appendix:omitted}
\begin{proof}[Proof of Lemma \ref{lemma:sub-main}]
	

		Consider $a_i$ and observe that:
		\begin{align}
		\frac{f(a_i|A_{i-1})}{c(a_i)}c(B)
		& = \sum_{e \in B} \frac{f(a_i|A_{i-1})}{c(a_i)}c(e)
		\nonumber
		\\ 	& \geq \sum_{e \in B} \frac{f(e|A_{i-1})}{c(e)}c(e)
		\label{ineq:main:cond}
		\\	& \geq f(B|A_{i-1})
		\label{ineq:main:sub}
		\\ 	& \geq f(B) - f(A_{i-1}).
		\label{ineq:main:mon}
		\end{align}
		Inequality \eqref{ineq:main:cond} follows from the condition in the lemma, inequality \eqref{ineq:main:sub} follows from the submodularity of $f$, and inequality \eqref{ineq:main:mon} follows from the monotonicity of $f$.
		Thus, from the above we can conclude that:
		$$
		f(B) - f(A_i)  \leq (f(B) - f(A_{i - 1}))
		\left(1 - \frac{c(a_i)}{c(B)}\right).
		$$
		Hence,
		$$
		f(B) - f(A_i)  \leq f(B) \prod_{j = 1}^{i}
		\left(1 - \frac{c(a_j)}{c(B)}\right).
		$$
		Applying the inequality $1 - x \leq e^{-x}$ we get that:
		$$
		f(B) - f(A_i)  \leq f(B)\cdot
		e^{-\frac{c(A_i)}{c(B)}}.
		$$
		Rearranging the terms and setting $i = k$ completes the proof.

	
	
	
\end{proof}





\end{document}
