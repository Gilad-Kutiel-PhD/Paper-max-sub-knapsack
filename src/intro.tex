In the \emph{Budgeted Maximum Cover} problem we are given a ground set of elements 
$X = \{x_1, \dots, x_n\}$ and a collection of subsets over $X$, 
$\mathcal{S} = \{S_1, \dots, S_m\}$.
A cost function, $c:\mathcal{S} \to \mathbb{R}_+$ assigns cost to each set 
and a weight function $w:X \to \mathbb{R}_+$ assigns weight to each element. 
The goal is to find a collection of sets $\mathcal{S'} \subseteq \mathcal{S}$ such that the 
total cost of the sets in the collection does not exceed a given budget, $B$, and the 
total weight of the elements covered by $\mathcal{S'}$ is maximal.
Khuller et. al \cite{khuller1999budgeted} 
give a $1-e^{-1}$-approximation algorithm for this problem and show that this is
the best possible unless P = NP.   

A set function $f$ is \emph{submodular} if $f(A \cap B) + f(A \cup B) \leq f(A) + f(B)$ 
for every two sets in the domain of the function. A set function, $f$, is \emph{monotone} if 
$f(A) \leq f(B)$ for every two sets, $A \subseteq B$, in the domain of the function.
Given a set of elements $U = \{e_1, \dots, e_n\}$, a cost function, 
$c:U \to \mathbb{R}_+$ and a monotone, submodular function, $f:2^U \to \mathbb{R}_+$ 
the goal in the \emph{Maximizing a Submodular Function Under
Knapsack Constraint} problem is to find a subset of $U$ with total cost that does not exceed
a given budget, $B$, with maximal value with respect to $f$.
This problem generalized the \emph{Budgeted Maximum Cover} problem. 
Sviridenko \cite{sviridenko2004note} showed that the same algorithm presented by 
Khuller et. al can be used to maximize general monotone submodular function 
with the same guarantee.
This algorithm requires $O(n^5)$ calls to the value oracle. 

A faster algorithm that runs in $O(n^2)$ time and achieves a $1 - e^{-1/2}$ approximation ratio
was also presented in \cite{khuller1999budgeted}. 
It was shown by by Krause and Guestrin \cite{krause2005note} that the same algorithm 
gives that same guarantee for a general monotone submodular function and requires only 
$O(n^2)$ calls to the value oracle.
In the above papers (and other papers as well), however, there is a flaw in the analysis of this
algorithm. 
In this paper we fix this flaw. 
We also consider a modification of this algorithm with the same running time and show that it
achieves a better approximation ratio.   